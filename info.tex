# Get from NullSafe Architect
# https://youtu.be/PJzIzytxJ2M
# https://www.youtube.com/watch?v=zkZPBHzvoMA
´´´ bash
-- USAR AIRFLOW
crear carpeta 
mkdir ETL_Final
cd ETL_Final
reiniciar terminal
# iniciar ambiente virtual en la carpeta deseada usando pyenv
$ virtualenv env
# activar ambiente virtual
source env/bin/activate
# instalar airflow en la carpeta, para eso traer la variable del home de airflow
# usar pwd para ver el path actual
pwd
# usar este comando y colocarle el path copiado de pwd
export AIRFLOW_HOME=/home/ygalvis/Documents/Study/ETL_Final
# crear las siguientes 3 variables:
# Si es necesario ver informacion de airflow usando AIRFLOW INFO
AIRFLOW_VERSION=2.10.5
#y colocar la version de python usada, este bash script traemos la version del environment actual:
PYTHON_VERSION="$(python --version | cut -d " " -f 2 | cut -d "." -f 1-2)"
# tarer las dependencias de instalación de airflow
CONSTRAINT_URL="https://raw.githubusercontent.com/apache/airflow/constraints-${AIRFLOW_VERSION}/constraints-${PYTHON_VERSION}.txt"
# instalar AIRFLOW
pip install "apache-airflow==${AIRFLOW_VERSION}" --constraint "${CONSTRAINT_URL}"

#iniciar airflow en modo standalone
airflow standalone
# para probar si tenemos fallas en el .py:
python /home/ygalvis/Documents/Study/ETL_Final/dags/example_dag.py

# probar un dag en modo debug
pip install apache-airflow[cncf.kubernetes]
airflow tasks test dag_extract get_csv 2025-01-01
´´´

# para ejecutar dags desde bash, para obtener el id de session puede usar el depurador de codigo de la página web, seleccionar el DAG y ver networks allí verá la sesion en los headers
curl -X POST -H 'accept: application/json' -H 'Content-type: application/json' -H 'Cookie: session=d27391ca-dd5c-46b5-8d24-d0518f99579c.3J0BM7QJlVX2Ef9URs44wjeVg9A' -d '{ "conf": {}, "dag_run_id": "xx2", "note": "string" }' http://127.0.0.1:7070/api/v1/dags/test11/dagRuns

Ahora accede a Metabase en http://localhost:3000
Cuando agregues PostgreSQL en Metabase, usa:

Host: metabase_db (no localhost)

Usuario: metabase

Contraseña: metabase_pass

Base de datos: metabase_db

Puerto: 5432

ir a carpeta de metabase y ejecutar:
java -jar metabase.jar

cambir base de daos a posgresql.

MB_DB_TYPE=postgres \
MB_DB_DBNAME=metabase_db \
MB_DB_PORT=5432 \
MB_DB_USER=postgres \
MB_DB_PASS=password \
MB_DB_HOST=localhost \
java -jar metabase.jar
